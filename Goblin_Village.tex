\documentclass[letterpaper,twocolumn,openany,nodeprecatedcode]{book}

% Use babel or polyglossia to automatically redefine macros for terms
% Armor Class, Level, etc...
% Default output is in English; captions are located in lib/dndstrings.sty.
% If no captions exist for a language, English will be used.
%1. To load a language with babel:
%	\usepackage[<lang>]{babel}
%2. To load a language with polyglossia:
%	\usepackage{polyglossia}
%	\setdefaultlanguage{<lang>}
\usepackage[english]{babel}
%\usepackage[italian]{babel}
% For further options (multilanguage documents, hypenations, language environments...)
% please refer to babel/polyglossia's documentation.

\usepackage[utf8]{inputenc}
\usepackage[singlelinecheck=false]{caption}
\usepackage{lipsum}
\usepackage{listings}
\usepackage{shortvrb}
\usepackage{stfloats}
\usepackage[layout=true]{dnd}

\captionsetup[table]{labelformat=empty,font={sf,sc,bf,},skip=0pt}

\MakeShortVerb{|}

\lstset{%
  basicstyle=\ttfamily,
  language=[LaTeX]{TeX},
  breaklines=true,
}

\title{Das Goblin Dorf\\
\large Eine schaukunst Goblinoider Kultur}
\author{von Jannik und Greg}
\date{27.05.2024}

\begin{document}

\frontmatter

\maketitle

\tableofcontents

\mainmatter%

\part{Hintergrundgeschichte}

\chapter{Einleitung}

\DndDropCapLine{D}{ieses Werk ist ein Kompendium} welches die Hintergrundgeschichte von Azazael und Metatron weiter beleuchten soll. Dabei wird beschrieben, wie das Dorf der beiden Gestalten zuvor war bis zum Zeitpunkt, als beide das Dorf verlassen haben. Ebenfalls wird in diesem Werk die verschiedensten Einrichtungen und NPC beschrieben, welche hier leben und Ihrem Werk nachgehen.

\section{Metatron}

Der Goblin Metatron ist ein kleingewachsenes Strassenkind, welches sich als Kind die Zeit damit vertrieb, krimskrams und nichtigkeiten von anderen zu klauen. Damit seine Opfer das Diebesgut nicht als das Ihre erkannten, verbastelte er oftmals seine Beute und kombinierte Sie zu kleinen Schaulustigkeiten oder Spielzeugen. Sein älterer Halbbruder Zaz, ein Hobgoblin, rügte Ihn oftmals, dass er von anderen Goblins nicht klauen soll.

Meta, aber, war ein kreatives Goblin kind, nicht jedes der Goblin kinder konnte aus Krimskrams, explodierenden Krimskrams bauen. Rauchenden Krimskrams. Krimskrams das ganz lauten Lärm machte. Krimskrams, welches Farbe auf den Köpfen unachtsamer Fußgänger ausschüttete. Meta konnte viel aus Krimskrams machen. Und das Tat er auch. Einer der Dorfältesten, Melchior, war stets bemüht um den jungen Knaben. Immerhin war er auch der jüngere Bruder des doch so mutigen Azazaels.

\subsection{Kindheit}
Als Metatron zum wiederholten male von seinem Bruder und den Dorfältesten zurechtgewiesen worden war, beschloss sich Meta keine Goblins zu beklauen, sondern vobeigehende Abenteurer. Wenn Zaz auf Patrolie war und einige Stunden nicht zuhause sein würde, schlich sich Metatron aus dem Dorf und begab sich in sein Geheimversteck. Eine kleine Höhle, etwa vier bis fünf Meter tief am unteren Ende einer Klippe. Diese Klippe ragte schätzungsweise acht bis zehn Meter in die Höhe. Der Eingang der kleinen Höhle war bedeckt von Dickicht und schützte Meta vor den Augen von vorbeiziehenden Bestien und Goblins. In diese Höhle, sammelte Meta alles, was er nicht zuhause aufbewahren konnte. Gestohlene Stofffetzen, Wagenräder, rostige Helme, Kaputte Schwerter mehrere Dolche, Eimer, Seil und vieles mehr. Metas wahrer Schatz aber war ein Buch, welches er einem schlafenden Abenteurer stahl. Darin fand Meta viele Zeichnungen und Erklärungen zu Arkanen Siegeln, Magische Artefakte und die Kunst Gegenstände zu verzaubern. Meta nutze jede freie Sekunde, in der Zaz kein Auge auf Ihn geworfen hatte, um das Buch zu studieren.

\subsection{Pubertät}
Als Metatron das Wehralter erreicht hatte, verpflichtete man Ihm Speer und Schild zu führen. Auch wenn Meta lieber die Zeit nutzen würde, sein geliebtes Zauberbuch zu studieren, gab er sich beim führen der Waffen alle Mühe. Nicht der Verteidigung des Dorfes willen und auch nicht weil die Ältesten Ihn dazu zwangen. Wenn Zaz Ihm den Umgang mit dem Speer beibrachte, fühlte er wie leidenschaftlich sein großer Bruder mit der Waffe umging. Wenn Meta die Waffe führte, spürte er, wie er nur die einzelnen Schrittte welche Ihm gezeigt worden waren nachähmte. Jeder einzelne Schritt war zwar deutlich durchgeführt worden, fühlten sich aber für Meta unnatürlich an. Er merkte wie er zwischen jedem Schritt, jeder Drehung, jedem Stellungswechsel nur einem klar festgelegtem Plan folgte. Dementsprechend sah seine Kampfkunst brüchig und unreif aus. Azazael hingegen, schien für Metatron das genaue Gegenteil zu sein

 Wenn er seinen großen Bruder betrachtete, schien sein Bruder gar keine vordefinierten Kampfschritte durchzuführen. Sein Umgang mit jeder Waffe schien eher dem eines Tänzers gleichzukommen. Besonders faszinierten Meta die Übungen seines Bruders, welche er alleine im Wald durchführte. Auch wenn sein Zaz glaubte dies in aller Heimlichkeit zu tun, wusste Meta stets wo sein Bruder aufzufinden war. Bei Gelegenheit beobachtete er sein Idol still und heimlich. Zugegebenermaßen, fand auch Meta einige Übungen Langweilig, wie die, in der sein Bruder in einer Eigenartigen Pose auf einem Stein nahezu Stundenlang regungslos verharrte und anscheinend nichts anderes tat als zu schlafen. Diese Momente nutze Meta um sein Buch weiterzulesen, nur um festzustellen, dass auch Stunden später sein Bruder, schweisüberströmt in seiner Eigenartigen Pose schlief. Am liebsten aber beobachtete er seinen Bruder beim Waffentraining. Zuerst dachte er, dass sein Bruder lediglich ein Stirnband trug, um sein Schweis aus den Augen zu halten. Später bemerkte er jedoch, dass sein Bruder gar nicht das Stirnband an seiner Stirn trug, sondern damit seine Augen verseckte. Bei dieser Übung, schienen die Bewegungen seines Bruders ganz anders als die Bewegungen, welche man Ihnen im Dorf beigebracht hatte. Als Meta endlich verstand was sein Bruder da tat, errötete Meta vor Peinlichkeit. Sein Bruder stellte sich anscheinend Gegner vor und parrierte Ihre Imaginären Angriffen aus. Plötzlich verstand Meta warum Zazs, sonst so flüssigen Bewegungen immer wieder mit unverständlichen sporadischen Bewegungen Komplimentiert wurden. Der nur halbausgeführte Ausfallschritt , wurde in Windeseile zurückgezogen, um sein Gewicht näher zum Boden zu verlagern. Warum? Anscheinend wich sein Bruder gerade einem Angriff gegen seinen Kopf aus. In windeseile, wurde aus dem Ausfallschritt eine ausweichende Seitwärtsrolle, die ohne zu zögern in einen Beinschwung überging. Zazs imaginärer Gegener musste am Boden liegen, denn bevor Metatron verstand wie sein Bruder eine so schnelle Rotation am Boden durchführen konnte, während dieser auf seiner Waffenhand lag, durchbohrte Zaz sein Ofper mit seinem Schwert. In diesem Moment wurde Meta klar, was für ein Genie sein Bruder war. Er musste der besste Kämpfer im Dorf, nein, auf der ganzen Welt sein. Und Meta würde genau so werden wie er.
 
 Doch Meta hatte für die Kampfkünste kein Talent. Sein Körper bewegte sich nicht wie er es wollte. Seine Hiebe und Stiche waren langsam und schwach. Geschickt war Meta mit seinen Händen, so konnte er beim Training manchmal sein Gegenüber entwaffnen indem er einige Bewegungen seines Bruders nachahmte, doch ständig wurde er von seinen Lehrern gerügt, dass er eine Ordentliche Form durchführen sollte. Die anderen Goblinoide lachten Ihn für seine schlechte Form und schwache Kunst aus. Lediglich sein Bruder sprach Ihm stets aufmunternde Worte zu.

\end{document}